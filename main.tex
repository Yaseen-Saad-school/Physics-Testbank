\documentclass[a4paper, 12pt]{extarticle}
\usepackage[
  top=0.5cm,
  bottom=1cm,
  left=1cm,
  right=1cm,
  headheight=17pt,
  includehead,includefoot,
  heightrounded,
]{geometry}
\usepackage{arabtex}
\usepackage{fancyhdr}
\usepackage{amsfonts}
\usepackage{cancel}
\usepackage{wasysym}
\usepackage{amssymb}
\usepackage{amsmath}
\usepackage{amsthm}
\usepackage{graphicx} 
\usepackage{siunitx} 
\usepackage{hyperref}
\usepackage{float}
\usepackage[compact, explicit]{titlesec}
\usepackage{xcolor}
\usepackage[T1]{fontenc}
\usepackage{listings}
\usepackage{color}
\usepackage{parskip}
\usepackage{graphicx}
\usepackage[utf8]{inputenc}
\usepackage{utf8}
\usepackage[T1]{fontenc}
\usepackage{adjustbox}
\usepackage{tikz}
\usepackage{cancel}
\usepackage{caption}
\usepackage{float}
\usepackage{subcaption}
\usepackage{listings}
\usepackage{xcolor}
\usepackage{lipsum}
\usepackage[most]{tcolorbox}
\usepackage{lmodern}
\newcommand{\uvec}[1]{\boldsymbol{\hat{\textbf{#1}}}}

\newtheoremstyle{solnss}
{}                % Space above
{}                % Space below
{\shape}          % Theorem body font % (default is "\upshape")
{}                % Indent amount
{\bfseries}       % Theorem head font % (default is \mdseries)
{.}               % Punctuation after theorem head % default: no punctuation
{ }               % Space after theorem head
{}                % Theorem head spec

\theoremstyle{definition}
% \newtheorem[]{soln}{Solution}[section]

\definecolor{xc5}{RGB}{0, 54, 154}

\pagestyle{fancy}
\fancyhead[L]{\includegraphics[width=20]{download.png} \raisebox{\height}{\centering STEM High School for Boys - 6th of October}}
\theoremstyle{definition}
\fancyhead[R]{\raisebox{\height}{\centering \textbf{\textcolor{black}{\textbf{Grade 11 Semester II Physics Test Bank}}} }}

\titleformat{\section}[block]
  {\bfseries\sffamily}{\\\Large\textcolor{xc5}{\S\thesection }\ \Large #1\\}
  {0em}{}

\titleformat{\subsection}[block]{\normalsize\bfseries\sffamily}{}{1pt}{#1 \textcolor{xc5}{Problem \thesubsection}:}

\newtcolorbox{prp}{colback=blue!5,enhanced,boxrule=0pt,leftrule=4pt,before skip= 5pt,colframe=xc5,skin=enhancedlast jigsaw,sharp corners}

\begin{document}
\setcode{utf8}

\include{title}

\addtocontents{toc}{\protect\setcounter{tocdepth}{1}}
\\[0.15cm]
{ \huge \bfseries Table of Contents: }


\normalsize\tableofcontents
\vspace{\fill}
\begin{center}
{\Large \<لَا تَنْسَوْنَا مِنْ صَالِحِ دُعَائِكُمْ> }    
\end{center}

\newpage

% Unit 8: Electromagnetic Induction (Ph.2.08)
\section{Electromagnetic Induction: Faraday's Law of Induction (Ph.2.08).}

\begin{prp}
\subsection{}
A circular loop in the plane of the paper lies in a 3.0 T magnetic field pointing into the paper. The loop’s diameter changes from 100 cm to 60 cm in 0.5 s. What is the magnitude of the average induced emf?  
\\  
a) 0.6 V \\  
b) 2.4 V \\  
c) 3 V \\  
d) 240 V  
\end{prp}

\begin{soln}
\textbf{Solution:}  
Using Faraday’s Law:  
\[
|\mathcal{E}| = \left|\frac{\Delta \Phi}{\Delta t}\right| = \frac{B \cdot \pi (r_1^2 - r_2^2)}{\Delta t} \approx 3 \, \text{V}.  
\]
\\ \textbf{Answer:} C- 3 V  
\end{soln}

\bigskip
\centerline{\rule{10cm}{0.4pt}}
\bigskip

\begin{prp}
\subsection{}
All the following devices depend on electromagnetic induction in their operation, except ...
\\  
a) The electric transformer in chargers \\  
b) The electric generator at power plants \\  
c) The electric motor in toys \\  
d) The metal detectors that people walk through at airports.  
\end{prp}

\begin{soln}
\textbf{Solution:}  
Electric motors operate via the motor effect, not induction.  \\
\\ \textbf{Answer:} C- The electric motor in toys.  
\end{soln}

\bigskip
\centerline{\rule{10cm}{0.4pt}}
\bigskip

\begin{prp}
\subsection{}
A wire moving in a magnetic field has NO induced voltage if ...  
\\  
a) It is moving along the field direction. \\  
b) It is made of copper. \\  
c) It is moving slowly. \\  
d) The wire is insulated.  
\end{prp}

\begin{soln} 

\textbf{Soution:} When the Wire is moving a long the field direction the angle between the direction of motion and the field lines is $0$, thus the induced e.m.f $\epsilon = Blv\sin{\theta} =  Blv\sin{0} =  Blv\cdot 0=  0$ \\
\\ \textbf{Answer:} a) It is moving along the field direction
\end{soln}

\bigskip
\centerline{\rule{10cm}{0.4pt}}
\bigskip

\begin{prp}
\subsection{}
A wire parallel to the paper moves downward into a flux density directed into the paper. The induced current will be ...  
\\  
a) Directed to the left \\  
b) Directed to the right \\  
c) Zero \\  
d) Opposite to the induced emf.  
\end{prp}

\begin{soln}
\textbf{Solution:}  
\\ \textbf{Answer:} b) Directed to the right 
\end{soln}

\bigskip
\centerline{\rule{10cm}{0.4pt}}
\bigskip

\begin{prp}
\subsection{}
The magnetic field strength inside a current-carrying coil will be greater if the coil encloses a...........  \\
A- Vacuum. \\ 
B- Wooden rod.  \\
C- Glass rod. \\
D- Rod of iron.
\end{prp}
\begin{soln}
\textbf{Solution:} \\
The magnetic field inside a coil is given by:
\[ B = \mu n I \]
where: \\
- $\mu$ is the permeability of the material inside the coil \\
- $n$ is the number of turns per unit length \\
- $I$ is the current

The permeability $\mu$ is much higher for ferromagnetic materials like iron ($\mu \gg \mu_0$) compared to vacuum, wood, or glass. Therefore, enclosing an iron rod significantly increases the magnetic field strength.

\\ \textbf{Answer:} D- Rod of iron.
\end{soln}

\bigskip
\centerline{\rule{10cm}{0.4pt}}
\bigskip
% Question 2
\begin{prp}
\subsection{}
A circular loop of wire is placed in a uniform magnetic field perpendicular to the magnetic lines. The strength of the magnetic field is B and the radius of the loop is R. What is the magnetic flux in the loop?  \\
A- $\dfrac{\pi B}{R^2}$ \\
B- $\pi BR^2$ \\
C- $\dfrac{\pi B}{R}$ \\
D- $\pi BR$
\end{prp}
\begin{soln}
\textbf{Solution:} \\
Magnetic flux $\Phi$ through a loop is given by:
\[ \Phi = B A \cos\theta \]
where: \\
- $B$ is the magnetic field strength \\
- $A$ is the area of the loop \\
- $\theta$ is the angle between B and the normal to the plane (0° here since perpendicular)

For a circular loop:
\[ A = \pi R^2 \]
\[ \Phi = B (\pi R^2) \cos 0° = \pi BR^2 \]

\\ \textbf{Answer:} B- $\pi BR^2$
\end{soln}



\bigskip
\centerline{\rule{10cm}{0.4pt}}
\bigskip

\begin{prp}
\subsection{}
What is the magnitude of the magnetic field at the core of a 120-turn solenoid of length 0.50 m carrying a current of 2.0 A? \\
A- $2.4 \times 10^{-4}$ T \\
B- $2.8 \times 10^{-5}$ T \\ 
C- $1.2 \times 10^{-6}$ T \\
D- $6.0 \times 10^{-4}$ T
\end{prp}
\begin{soln}
\textbf{Solution:} \\
The magnetic field inside a solenoid is given by: \\
\[ B = \mu_0 n I \]
where: \\
- $\mu_0 = 4\pi \times 10^{-7}$ Tm/A \\
- $n = \frac{120}{0.50} = 240$ turns/m \\
- $I = 2.0$ A \\

Calculating: 
\[ B = (4\pi \times 10^{-7})(240)(2.0) \approx 6.0 \times 10^{-4} \text{ T} \] 
\\ \textbf{Answer:} D- $6.0 \times 10^{-4}$ T
\end{soln}

\bigskip
\centerline{\rule{10cm}{0.4pt}}
\bigskip

\begin{prp}
\subsection{}
Solenoid A has length L and N turns, solenoid B has length 2L and N turns, and solenoid C has length L/2 and 2N turns. If each solenoid carries the same current, rank by the strength of the magnetic field in the center of each solenoid from largest to smallest. \\
A- A, B, C \\ 
B- A, C, B \\ 
C- B, C, A \\
D- C, A, B
\end{prp}
\begin{soln}
\textbf{Solution:} \\
Magnetic field strength depends on turn density ($n = N/L$): \\
- Solenoid A: $n_A = \dfrac{N}{L}$ \\
- Solenoid B: $n_B = \dfrac{N}{2L} = \dfrac{1}{2}n_A$ \\ 
- Solenoid C: $n_C = \dfrac{2N}{\dfrac{L}{2}} = 4n_A$ \\

Field ranking follows turn density: C > A > B \\

\\ \textbf{Answer:} D- C, A, B
\end{soln}

\bigskip
\centerline{\rule{10cm}{0.4pt}}
\bigskip

\begin{prp}
\subsection{}
A magnet attracts a piece of iron. The iron can then attract another piece of iron. On the basis of domain alignment, the explanation is... \\
A- The magnet induces magnetic poles in the 2nd piece then attracts the 1st \\
B- The magnet induces a current in the 1st piece. \\
C- The magnet causes domain alignment in the iron. \\
D- The magnet induces a current in the 2nd piece.
\end{prp}
\begin{soln}
\textbf{Solution:} \\
The correct explanation involves magnetic domain alignment: \\
1. External magnet aligns domains in 1st iron piece \\
2. Aligned domains make 1st piece temporarily magnetic \\
3. 1st piece can then align domains in 2nd piece \\

\\ \textbf{Answer:} C- The magnet causes domain alignment in the iron.
\end{soln}

\bigskip
\centerline{\rule{10cm}{0.4pt}}
\bigskip

\begin{prp}
\subsection{}
A circular coil (diameter 10 cm) produces $5 \times 10^{-5}$ T at center. If stretched to 20 cm solenoid, the new flux density is: \\
A- $2.5 \times 10^{-5}$ T \\
B- $5 \times 10^{-5}$ T \\
C- $1 \times 10^{-5}$ T \\
D- $10 \times 10^{-5}$ T
\end{prp}
\begin{soln}
\textbf{Solution:} \\
Key principle: Flux conservation. \\
Original coil field: $B_{\text{coil}} = N\dfrac{\mu_0 I}{2R} =N\dfrac{\mu_0 I}{0.10} = 5 \times 10^{-5}$ T \\
Solenoid field: $B_{\text{solenoid}} = N\dfrac{\mu_0 I}{L} =N\dfrac{\mu_0 I}{0.2} = N\dfrac{\mu_0 I}{0.10}\cdot\dfrac{1}{2} =  2.5 \times 10^{-5}$ T \\
For same current and total turns, field reduces by geometry factor. \\

\\ \textbf{Answer:} A- $2.5 \times 10^{-5}$ T
\end{soln}

\bigskip
\centerline{\rule{10cm}{0.4pt}}
\bigskip

\begin{prp}
\subsection{}
A long wire carries 1 A current. Magnetic field 2 m away is: \\
A- $1 \times 10^{-3}$ T \\
B- $1 \times 10^{-4}$ T \\
C- $1 \times 10^{-6}$ T \\
D- $1 \times 10^{-7}$ T
\end{prp}
\begin{soln}
\textbf{Solution:} \\
Field from infinite wire: \\
\[ B = \frac{\mu_0 I}{2\pi r} = \frac{(4\pi \times 10^{-7})(1)}{2\pi \times 2} = 1 \times 10^{-7} \text{ T} \] \\

\\ \textbf{Answer:} D- $1 \times 10^{-7}$ T
\end{soln}




\bigskip
\centerline{\rule{10cm}{0.4pt}}
\bigskip

\begin{prp}
\subsection{}
A coil of 100 turns with face area 20 cm$^2$ is placed perpendicular to a uniform magnetic field (0.2 T). If the field direction reverses in 0.2 s, the average induced emf is: \\
A- 0.2 V \\
B- 0.4 V \\
C- 0.6 V \\
D- 0.8 V
\end{prp}
\begin{soln}
\textbf{Solution:} \\
Using Faraday's Law: \\
\[ \epsilon = -N \frac{\Delta \Phi}{\Delta t} \] \\
\[ \Delta \Phi = BA\cos\theta = (0.2)(20 \times 10^{-4})(\cos 0° - \cos 180°) = 8 \times 10^{-4} \text{ Wb} \] \\
\[ \epsilon = -100 \left( \frac{8 \times 10^{-4}}{0.2} \right) = 0.4 \text{ V} \] \\

\\ \textbf{Answer:} B- 0.4 V
\end{soln}

\bigskip
\centerline{\rule{10cm}{0.4pt}}
\bigskip
\begin{prp}
\subsection{}
A circular coil (1 turn, 22 cm radius) perpendicular to 0.05 T field is turned 90° in 0.25 s. Average emf generated is: \\
A- 0.03 V \\
B- 0.04 V \\
C- 0.3 V \\
D- 0.4 V
\end{prp}
\begin{soln}
\textbf{Solution:} \\
\[ \epsilon = -\frac{\Delta \Phi}{\Delta t} = -\frac{BA(\cos 90° - \cos 0°)}{0.25} \] \\
\[ A = \pi (0.22)^2 = 0.152 \text{ m}^2 \] \\
\[ \epsilon = \frac{(0.05)(0.152)(1)}{0.25} = 0.03 \text{ V} \] \\

\\ \textbf{Answer:} A- 0.03 V
\end{soln}

\bigskip
\centerline{\rule{10cm}{0.4pt}}
\bigskip

\begin{prp}
\subsection{}
Solenoid (400 turns, 4 cm$^2$ each) parallel to 0.3 T field. When B increases to 0.5 T in 2 ms: \\
A- 16 V \\
B- 8 V \\
C- 8 V \\
D- 16 V
\end{prp}
\begin{soln}
\textbf{Solution:} \\
\[ \epsilon = -N \frac{\Delta B A}{\Delta t} = -400 \left( \frac{0.2 \times 4 \times 10^{-4}}{0.002} \right) = 16 \text{ V} \] \\

\\ \textbf{Answer:} A- 16 V
\end{soln}

\bigskip
\centerline{\rule{10cm}{0.4pt}}
\bigskip

\begin{prp}
\subsection{}
Same solenoid, B decreases to 0.2 T in 2 ms: \\
A- 6 V \\
B- 8 V \\
C- 10 V \\
D- 12 V
\end{prp}
\begin{soln}
\textbf{Solution:} \\
\[ \epsilon = -400 \left( \frac{-0.1 \times 4 \times 10^{-4}}{0.002} \right) = 8 \text{ V} \] \\

\\ \textbf{Answer:} B- 8 V
\end{soln}

\bigskip
\centerline{\rule{10cm}{0.4pt}}
\bigskip

\begin{prp}
\subsection{}
Rectangular coil (100 turns, 20 cm × 10 cm) perpendicular to field, turned 90° in 0.2 s generates 0.4 V. Find B: \\
A- 0.01 T \\
B- 0.03 T \\
C- 0.04 T \\
D- 0.05 T
\end{prp}
\begin{soln}
\textbf{Solution:} \\
\[ \epsilon = -N \frac{BA(\cos 90° - \cos 0°)}{\Delta t} \] \\
\[ 0.4 = -100 \left( \frac{-B(0.02)}{0.2} \right) \Rightarrow B = 0.04 \text{ T} \] \\

\\ \textbf{Answer:} C- 0.04 T
\end{soln}

\bigskip
\centerline{\rule{10cm}{0.4pt}}
\bigskip

\begin{prp}
\subsection{}
Coil (400 turns, 50 cm$^2$) with perpendicular 0.2 T field. If flux vanishes in 0.01 s: \\
A- 20 V \\
B- 40 V \\
C- 60 V \\
D- 80 V
\end{prp}
\begin{soln}
\textbf{Solution:} \\
\[ \epsilon = -400 \left( \frac{0 - 0.2 \times 50 \times 10^{-4}}{0.01} \right) = 40 \text{ V} \] \\

\\ \textbf{Answer:} B- 40 V
\end{soln}

\bigskip
\centerline{\rule{10cm}{0.4pt}}
\bigskip

\begin{prp}
\subsection{}
30 cm copper rod moves at 0.5 m/s in 0.8 T field. Induced emf when: \\
1. Perpendicular to field \\
2. Parallel to field \\
A- 0.12 V | 0.12 V \\
B- 0.12 V | 0 V \\
C- 0 V | 0.12 V \\
D- 0 V | 0 V
\end{prp}
\begin{soln}
\textbf{Solution:} \\
\[ \epsilon = B l v \sin\theta \] \\
1. $\theta = 90° \Rightarrow 0.8 \times 0.3 \times 0.5 = 0.12$ V \\
2. $\theta = 0° \Rightarrow 0$ V \\

\\ \textbf{Answer:} B- 0.12 V | 0 V
\end{soln}

\bigskip
\centerline{\rule{10cm}{0.4pt}}
\bigskip

\begin{prp}
\subsection{}
Wire (0.5 m) moves at 2 m/s in 0.4 T field, producing 0.336 V. Find angle between v and B: \\
A- 36° \\
B- 57° \\
C- 64° \\
D- 82°
\end{prp}
\begin{soln}
\textbf{Solution:} \\
\[ \epsilon = B l v \sin\theta \] \\
\[ 0.336 = 0.4 \times 0.5 \times 2 \sin\theta \Rightarrow \theta \approx 57° \] \\

\\ \textbf{Answer:} B- 57°
\end{soln}

\bigskip
\centerline{\rule{10cm}{0.4pt}}
\bigskip

\begin{prp}
\subsection{}
The electric current of an electromagnet is switched off. The magnetic property will: \\
A- Remain as it is \\
B- Become zero \\
C- Decrease with time for long \\
D- Increase with time for long
\end{prp}
\begin{soln}
\textbf{Solution:} \\
An electromagnet requires continuous current to maintain its magnetic field. When current is switched off: \\
- Magnetic domains randomize immediately \\
- No residual magnetism (unlike permanent magnets) \\

\\ \textbf{Answer:} B- Become zero
\end{soln}

\bigskip
\centerline{\rule{10cm}{0.4pt}}
\bigskip

\begin{prp}
\subsection{}
A wire (0.4 m) moves perpendicular to 0.7 T field, generating 1 V emf. Find velocity: \\

A- 1.79 m/s \\
B- 3.57 m/s \\
C- 7.14 m/s \\
D- 8.32 m/s
\end{prp}
\begin{soln}
\textbf{Solution:} \\
\[ \epsilon = B l v \] \\
\[ v = \frac{\epsilon}{Bl} = \frac{1}{0.7 \times 0.4} = 3.57 \text{ m/s} \] \\

\\ \textbf{Answer:} B- 3.57 m/s
\end{soln}

\bigskip
\centerline{\rule{10cm}{0.4pt}}
\bigskip

\begin{prp}
\subsection{}
Which law states: "Induced EMF equals time rate of change of magnetic flux"? \\
. \\
A. Faraday's law \\
B. Coulomb's law \\
C. Kirchhoff's laws \\
D. Laplace's law 
\end{prp}
\begin{soln}
\textbf{Solution:} \\
This is the fundamental statement of Faraday's Law of Induction: \\
\[ \epsilon = -\frac{\Delta\Phi_B}{\Delta t} \] \\

\\ \textbf{Answer:} A. Faraday's law
\end{soln}

\bigskip
\centerline{\rule{10cm}{0.4pt}}
\bigskip

\begin{prp}
\subsection{}
Circular loop (B=0.125 T) shrinks at 2 mm/s. Find emf when r=4 cm: \\

A- $0.52\pi \mu$V \\
B- $20\pi \mu$V \\
C- $\dfrac{2}{3} \mu$V \\
D- $\dfrac{3\pi}{2} \mu$V \\
\end{prp}
\begin{soln}
\textbf{Solution:} \\
\[ \epsilon = \left| \frac{\Delta\Phi}{\Delta t} \right| = B \left| \frac{\Delta A}{ \Delta t} \right| = B \cdot 2\pi r \left| \frac{\Delta r}{\Delta t} \right| \] \\
\[ = 0.125 \times 2\pi \times 0.04 \times 0.002 = 20\pi \times 10^{-6} \text{V} \] \\

\\ \textbf{Answer:} 2. $20\pi \mu$V
\end{soln}

\bigskip
\centerline{\rule{10cm}{0.4pt}}
\bigskip

\begin{prp}
\subsection{}
Self-inductance per unit length of solenoid (radius R, n turns/m): \\
. \\
1. $\mathcal{L} = \mu_0 n \pi R^2$ \\
2. $\mathcal{L} = \mu_0 n^2 \pi R^2$ \\
3. $\mathcal{L} = \mu_0 n^2 \pi R$ \\
4. $\mathcal{L} = \mu_0 n \pi R$ \\
5. $\mathcal{L} = \mu_0 n \pi R / 2$
\end{prp}
\begin{soln}
\textbf{Solution:} \\
For a solenoid: \\
\[ \mathcal{L} = \mu_0 n^2 A = \mu_0 n^2 (\pi R^2) \] \\
Per unit length: \\
\[ \mathcal{L}_{\text{unit length}} = \mu_0 n^2 \pi R^2 \] \\

\\ \textbf{Answer:} 2. $\mathcal{L} = \mu_0 n^2 \pi R^2$
\end{soln}

\bigskip
\centerline{\rule{10cm}{0.4pt}}
\bigskip

% Unit 9: Inductive Elements in DC Circuits (Ph.2.09)
\section{Back-EMF, Energy Storage, and RL Circuits (Ph.2.09).}

\begin{prp}
\subsection{}
Current in a coil is changing at a rate of 100 A/s. If the emf developed in the coil is 0.5 V, what is the self-inductance of the coil?  
.  
\\ a) 2 Henry \\  
b) 5 milli-Henry \\  
c) 50 Henry \\  
d) 200 Henry  
\end{prp}

\begin{soln}
\textbf{Solution:}  
\[
L = \frac{\mathcal{E}}{|\Delta I/\Delta t|} = \frac{0.5}{100} = 0.005 \, \text{H} = 5 \, \text{mH}.  
\]
\\ \textbf{Answer:} b) 5 milli-Henry.  
\end{soln}

\bigskip
\centerline{\rule{10cm}{0.4pt}}
\bigskip

\begin{prp}
\subsection{}
Two solenoids X and Y have the same length and diameter. The number of turns in coil X is twice those in coil Y. Self-inductance of coil X relative to coil Y is ...  
.  
\\ a) Exactly the same \\  
b) About half \\  
c) Two times greater \\  
d) Four times greater  
\end{prp}

\begin{soln}
\textbf{Solution:}  
\( L \propto N^2 \). Doubling \( N \) quadruples \( L \).  
\\ \textbf{Answer:} d) Approximately four times greater.  
\end{soln}

\bigskip
\centerline{\rule{10cm}{0.4pt}}
\bigskip

\begin{prp}
\subsection{}
The reverse emf induced across the motor coil during rotation ...   \\  
\\ a) Keeps the coil rotating in one direction. \\  
b) Makes the coil rotate at uniform speed. \\  
c) Makes the motor more powerful. \\  
d) Gives a push when the coil is perpendicular.  
\end{prp}

\begin{soln}
\textbf{Solution:}  
Back EMF opposes current, stabilizing speed.  
\\ \textbf{Answer:} b) Makes the coil rotate at a uniform speed.  
\end{soln}

\bigskip
\centerline{\rule{10cm}{0.4pt}}
\bigskip

\begin{prp}
\subsection{}
After DC current flows in an inductor, its intensity becomes constant due to: \\

A- Generation of forward currents \\
B- Generation of eddy currents \\
C- Vanishing of self induction \\
D- Presence of back currents
\end{prp}
\begin{soln}
\textbf{Solution:} \\
When current stabilizes: \\
- Back EMF ($\epsilon = -L\frac{\Delta I}{\Delta t}$) vanishes as $\frac{\Delta I}{\Delta t} = 0$ \\
\\ \textbf{Answer:} C- Vanishing of self induction
\end{soln}

\bigskip
\centerline{\rule{10cm}{0.4pt}}
\bigskip


\begin{prp}
\subsection{}
Slow current growth in a solenoid at circuit closing is due to: \\
. \\
a) Forward induced current \\
b) Back induced EMF \\
c) Energy loss in coil \\
d) Increased ohmic resistance
\end{prp}
\begin{soln}
\textbf{Solution:} \\
Lenz's Law: \\ 
\[ \epsilon_{\text{back}} = -L\frac{\Delta I}{\Delta t} \text{ opposes current change} \] \\
\\ \textbf{Answer:} b) Back induced EMF
\end{soln}

\bigskip
\centerline{\rule{10cm}{0.4pt}}
\bigskip


\begin{prp}
\subsection{}
If turns (N) and length ($\ell$) of inductor double (A constant), new $L$ is: \\

A- $\frac{L}{2}$ \\
B- $L$ \\
C- $2L$ \\
D- $4L$
\end{prp}
\begin{soln}
\textbf{Solution:} \\
\[ L \propto \frac{N^2}{\ell} \Rightarrow \frac{(2N)^2}{2\ell} = \frac{4}{2}L = 2L \] \\
\\ \textbf{Answer:} C- $2L$
\end{soln}

\bigskip
\centerline{\rule{10cm}{0.4pt}}
\bigskip


\begin{prp}
\subsection{}
To double solenoid's $L$ (other factors constant): \\
A- Double turns \\
B- Double length \\
C- Double cross-section \\
D- Remove iron core
\end{prp}
\begin{soln}
\textbf{Solution:} \\
\[ L = \mu_0 \mu_r \frac{N^2 A}{\ell} \] \\
Doubling $N$: $L$ becomes $4\times$ (not $2\times$) \\
Doubling $A$: Directly doubles $L$ \\
\\ \textbf{Answer:} C- Double cross-sectional area
\end{soln}

\bigskip
\centerline{\rule{10cm}{0.4pt}}
\bigskip


\begin{prp}
\subsection{}
Coil ($L=0.1$ H) with iron core will have: \\
A- $L=0.1$ H \\
B- $L>0.1$ H \\
C- $L<0.1$ H \\
D- Depends on AC current
\end{prp}
\begin{soln}
\textbf{Solution:} \\
Iron core increases permeability ($\mu_r \gg 1$): \\
\[ L_{\text{new}} = \mu_r L_{\text{air}} \] \\
\\ \textbf{Answer:} B- Greater than 0.1 H
\end{soln}

\bigskip
\centerline{\rule{10cm}{0.4pt}}
\bigskip


\begin{prp}
\subsection{}
Inductor ($L=1$ H, $I=2$ A) produces $4\times10^{-3}$ Wb flux. Find $N$: \\
A- 200 \\
B- 300 \\
C- 500 \\
D- 1000
\end{prp}
\begin{soln}
\textbf{Solution:} \\
\[ \Phi = LI \Rightarrow N = \frac{LI}{\Phi} = \frac{1 \times 2}{4 \times 10^{-3}} = 500 \] \\
\\ \textbf{Answer:} C- 500 turns
\end{soln}

\bigskip
\centerline{\rule{10cm}{0.4pt}}
\bigskip


\begin{prp}
\subsection{}
Solenoid (100 turns, $\Delta I=4$ A, $\Delta \Phi=0.05$ Wb) has $L$: \\
A- 4 H \\
B- 0.5 H \\
C- 1.25 H \\
D- 0.2 H
\end{prp}
\begin{soln}
\textbf{Solution:} \\
\[ L = N \frac{\Delta \Phi}{\Delta I} = 100 \times \frac{0.05}{4} = 1.25 \text{ H} \] \\
\\ \textbf{Answer:} C- 1.25 H
\end{soln}

\bigskip
\centerline{\rule{10cm}{0.4pt}}
\bigskip


\begin{prp}
\subsection{}
Coil ($L=0.03$ H, $\Phi=6\times10^{-4}$ Wb) current vanishes in 0.02 s: \\
(i) Induced EMF: \\
A- 3 V \\
B- 6 V \\
C- 9 V \\
D- 12 V
\end{prp}
\begin{soln}
\textbf{Solution:} \\
\[ \epsilon = L \frac{\Delta I}{\Delta t} = \frac{N \Delta \Phi}{\Delta t} \] \\
First find $I_0$: \\
\[ \Phi = LI/N \Rightarrow I_0 = \frac{N\Phi}{L} = \frac{100 \times 6 \times 10^{-4}}{0.03} = 2 \text{ A} \] \\
Then: \\
\[ \epsilon = 0.03 \times \frac{2}{0.02} = 3 \text{ V} \] \\
\\ \textbf{Answer:} A- 3 V
\end{soln}

\bigskip
\centerline{\rule{10cm}{0.4pt}}
\bigskip


\begin{prp}
\subsection{}
A 2 H inductor is connected in series with a 10 Ω resistor and a 12 V DC source. At the instant the circuit is closed: \\
(a) What is the initial current? \\
(b) What is the initial voltage across the inductor? \\
(c) After a long time, what is the steady-state current? \\
. \\
1. (a) 0 A, (b) 12 V, (c) 1.2 A \\
2. (a) 1.2 A, (b) 0 V, (c) 1.2 A \\
3. (a) 0 A, (b) 0 V, (c) 1.2 A \\
4. (a) 0 A, (b) 12 V, (c) 0 A
\end{prp}
\begin{soln}
\textbf{Solution:} \\
(a) At t=0, inductor opposes current change $\Rightarrow I(0) = 0$ A \\
(b) All voltage drops across inductor initially $\Rightarrow V_L(0) = 12$ V \\
(c) At steady-state, inductor acts as wire $\Rightarrow I = V/R = 12/10 = 1.2$ A \\
\\ \textbf{Answer:} 1. (a) 0 A, (b) 12 V, (c) 1.2 A
\end{soln}

\bigskip
\centerline{\rule{10cm}{0.4pt}}
\bigskip

\begin{prp}
\subsection{}
An RL circuit has $R = 5 \Omega$ and $L = 10$ H. The time constant ($\tau$) is: \\
. \\
1. 0.5 s \\
2. 2 s \\
3. 50 s \\
4. 0.2 s
\end{prp}
\begin{soln}
\textbf{Solution:} \\
\[ \tau = \frac{L}{R} = \frac{10}{5} = 2 \text{ s} \] \\
\\ \textbf{Answer:} 2. 2 s
\end{soln}

\bigskip
\centerline{\rule{10cm}{0.4pt}}
\bigskip

\begin{prp}
\subsection{}
When current through a 0.5 H inductor decreases from 4 A to 0 A in 0.1 s, the induced back EMF is: \\
. \\
1. 20 V, opposing current decrease \\
2. 20 V, aiding current decrease \\
3. 0.02 V, opposing current decrease \\
4. 0.02 V, aiding current decrease
\end{prp}
\begin{soln}
\textbf{Solution:} \\
\[ \epsilon = -L \frac{\Delta I}{\Delta t} = -0.5 \left(\frac{0-4}{0.1}\right) = 20 \text{ V} \] \\
Lenz's Law: opposes current decrease \\
\\ \textbf{Answer:} 1. 20 V, opposing current decrease
\end{soln}

\bigskip
\centerline{\rule{10cm}{0.4pt}}
\bigskip

\begin{prp}
\subsection{}
A 4 H inductor carries a steady current of 3 A. The energy stored is: \\
. \\
1. 6 J \\
2. 12 J \\
3. 18 J \\
4. 36 J
\end{prp}
\begin{soln}
\textbf{Solution:} \\
\[ U = \frac{1}{2}LI^2 = 0.5 \times 4 \times 3^2 = 18 \text{ J} \] \\
\\ \textbf{Answer:} 3. 18 J
\end{soln}

\bigskip
\centerline{\rule{10cm}{0.4pt}}
\bigskip

\begin{prp}
\subsection{}
The current $I(t)$ in an RL circuit during charging follows: \\
. \\
1. Linear increase \\
2. Exponential decay \\
3. Exponential growth \\
4. Step function
\end{prp}
\begin{soln}
\textbf{Solution:} \\
\[ I(t) = I_{\text{max}}(1 - e^{-t/\tau}) \text{ (exponential growth)} \] \\
\\ \textbf{Answer:} 3. Exponential growth
\end{soln}

\bigskip
\centerline{\rule{10cm}{0.4pt}}
\bigskip


\begin{prp}
\subsection{}
Two inductors (3 H and 6 H) in series have equivalent inductance: \\
. \\
1. 2 H \\
2. 3 H \\
3. 6 H \\
4. 9 H
\end{prp}
\begin{soln}
\textbf{Solution:} \\
\[ L_{\text{eq}} = L_1 + L_2 = 3 + 6 = 9 \text{ H} \] \\
\\ \textbf{Answer:} 4. 9 H
\end{soln}

\bigskip
\centerline{\rule{10cm}{0.4pt}}
\bigskip

\begin{prp}
\subsection{}
In an RL circuit with $\tau = 4$ s, time to reach 63\% of max current is: \\
. \\
1. 1 s \\
2. 4 s \\
3. 6 s \\
4. 8 s
\end{prp}
\begin{soln}
\textbf{Solution:} \\
By definition, $\tau$ is time to reach $1-e^{-1} \approx 63\%$ \\
\\ \textbf{Answer:} 2. 4 s
\end{soln}

\bigskip
\centerline{\rule{10cm}{0.4pt}}
\bigskip

\begin{prp}
\subsection{}
During inductor discharge, current $I(t)$: \\
. \\
1. Drops linearly to zero \\
2. Follows $I_0 e^{-t/\tau}$ \\
3. Remains constant \\
4. Oscillates
\end{prp}
\begin{soln}
\textbf{Solution:} \\
\[ I(t) = I_0 e^{-t/\tau} \text{ (exponential decay)} \] \\
\\ \textbf{Answer:} 2. Follows $I_0 e^{-t/\tau}$
\end{soln}

\bigskip
\centerline{\rule{10cm}{0.4pt}}
\bigskip

\begin{prp}
\subsection{}
Why are iron-core inductors used in power supplies? \\
. \\
1. Reduce energy storage \\
2. Increase $L$ via higher $\mu_r$ \\
3. Eliminate back EMF \\
4. Decrease resistance
\end{prp}
\begin{soln}
\textbf{Solution:} \\
Iron increases permeability ($\mu_r \gg 1$), boosting $L = \mu_0 \mu_r N^2 A/\ell$ \\
\\ \textbf{Answer:} 2. Increase $L$ via higher $\mu_r$
\end{soln}

\bigskip
\centerline{\rule{10cm}{0.4pt}}
\bigskip

\begin{prp}
\subsection{}
In an RL circuit, voltage across resistor $V_R$ during charging: \\
. \\
A- Starts at 0 and rises exponentially \\
B- Starts at $V_{\text{source}}$ and decays exponentially \\
C- Is constant \\
D- Oscillates
\end{prp}
\begin{soln}
\textbf{Solution:} \\
\[ V_R(t) = I(t)R = V_{\text{source}}(1-e^{-t/\tau}) \text{ (exponential rise)} \] \\
\\ \textbf{Answer:} A- Starts at 0 and rises exponentially
\end{soln}

\bigskip
\centerline{\rule{10cm}{0.4pt}}
\bigskip

\begin{prp}
\subsection{}
Two inductors of 4 H and 6 H are connected in series. What is the equivalent inductance? \\
A- 2 H \\
B- 4 H \\
C- 6 H \\
D- 10 H
\end{prp}

\begin{soln}
\textbf{Solution:} \\
For series connection: $ L_{eq} = L_1 + L_2 = 4 + 6 = 10 \text{ H} $ \\
\\ \textbf{Answer:} D- 10 H
\end{soln}

\bigskip
\centerline{\rule{10cm}{0.4pt}}
\bigskip

\begin{prp}
\subsection{}
Two inductors of 4 H and 6 H are connected in parallel. What is the equivalent inductance? \\
A- 2 H \\
B- 2.4 H \\
C- 4 H \\
D- 6 H
\end{prp}
\begin{soln}
\textbf{Solution:} \\
For parallel connection: 
$$ \frac{1}{L_{eq}} = \frac{1}{L_1} + \frac{1}{L_2} = \frac{1}{4} + \frac{1}{6} = \frac{5}{12} $$ \\
$$ \Rightarrow L_{eq} = \frac{12}{5} = 2.4 \text{ H} $$ \\
\\ \textbf{Answer:} B- 2.4 H
\end{soln}

\bigskip
\centerline{\rule{10cm}{0.4pt}}
\bigskip

\begin{prp}
\subsection{}
An RL circuit contains a 10 Ω resistor and a 2 H inductor. What is the time constant of the circuit? \\
A- 0.2 s \\
B- 0.5 s \\
C- 2 s \\
D- 5 s
\end{prp}
\begin{soln}
\textbf{Solution:} \\
Time constant: 
$$ \tau = \frac{L}{R} = \frac{2}{10} = 0.2 \text{ s} $$ \\
\\ \textbf{Answer:} A- 0.2 s
\end{soln}

\bigskip
\centerline{\rule{10cm}{0.4pt}}
\bigskip

\begin{prp}
\subsection{}
A 0.8 H inductor experiences a current change from 4 A to 0 A in 0.02 seconds. What is the induced back EMF? \\
A- 40 V \\
B- 80 V \\
C- 160 V \\
D- 200 V
\end{prp}
\begin{soln}
\textbf{Solution:} \\
Using:
$$ \epsilon = -L \frac{\Delta I}{\Delta t} = -0.8 \times \frac{(0 - 4)}{0.02} = 160 \text{ V} $$ \\
\\ \textbf{Answer:} C- 160 V
\end{soln}

\bigskip
\centerline{\rule{10cm}{0.4pt}}
\bigskip

\begin{prp}
\subsection{}
In an RL circuit with a 12 V battery, 3 Ω resistor, and 1 H inductor, what is the steady-state current? \\
A- 0 A \\
B- 3 A \\
C- 4 A \\
D- 12 A
\end{prp}
\begin{soln}
\textbf{Solution:} \\
At steady state: inductor acts as short, so:
$$ I = \frac{V}{R} = \frac{12}{3} = 4 \text{ A} $$ \\
\\ \textbf{Answer:} C- 4 A
\end{soln}

\bigskip
\centerline{\rule{10cm}{0.4pt}}
\bigskip

\begin{prp}
\subsection{}
During charging in an RL circuit, what happens to the voltage across the inductor over time? \\
A- It increases linearly \\
B- It decreases exponentially \\
C- It stays constant \\
D- It oscillates sinusoidally
\end{prp}
\begin{soln}
\textbf{Solution:} \\
Initial voltage is high (equals supply), then decays exponentially as current builds up. \\
\\ \textbf{Answer:} B- It decreases exponentially
\end{soln}

\bigskip
\centerline{\rule{10cm}{0.4pt}}
\bigskip

\begin{prp}
\subsection{}
A 2 H inductor carries a steady current of 5 A. How much energy is stored in its magnetic field? \\

A- 5 J \\
B- 10 J \\
C- 25 J \\
D- 50 J
\end{prp}
\begin{soln}
\textbf{Solution:} \\
Energy stored:
$$ U = \frac{1}{2} L I^2 = \frac{1}{2} \cdot 2 \cdot 5^2 = 25 \text{ J} $$ \\
\\ \textbf{Answer:} C- 25 J
\end{soln}

\bigskip
\centerline{\rule{10cm}{0.4pt}}
\bigskip

\begin{prp}
\subsection{}
When an inductor discharges through a resistor, how does the current behave over time? \\
A- Linear decay to zero \\
B- Exponential growth to max value \\
C- Exponential decay to zero \\
D- Oscillatory behavior
\end{prp}
\begin{soln}
\textbf{Solution:} \\
Discharging follows exponential decay:
$$ I(t) = I_0 e^{-t/\tau} $$ \\
\\ \textbf{Answer:} C- Exponential decay to zero
\end{soln}

\bigskip
\centerline{\rule{10cm}{0.4pt}}
\bigskip

\begin{prp}
\subsection{}
Which graph best represents the voltage across an inductor during discharge in an RL circuit? \\
A- Horizontal line at maximum voltage \\
B- Straight line decreasing to zero \\
C- Exponential curve falling to zero \\
D- Sine wave
\end{prp}
\begin{soln}
\textbf{Solution:} \\
Voltage across inductor decays exponentially as current drops. \\
\\ \textbf{Answer:} C- Exponential curve falling to zero
\end{soln}

\bigskip
\centerline{\rule{10cm}{0.4pt}}
\bigskip

\begin{prp}
\subsection{}
If resistance in an RL circuit is doubled while keeping inductance constant, what happens to the time constant? \\
A- Doubles \\
B- Halves \\
C- Remains the same \\
D- Quadruples
\end{prp}
\begin{soln}
\textbf{Solution:} \\
Since $ \tau = \frac{L}{R} $, doubling R halves the time constant. \\
\\ \textbf{Answer:} B- Halves
\end{soln}
% Unit 10: Energy Transmission (Ph.2.10)
\section{Production and Transmission of Electrical Energy (Ph.2.10).}

\begin{prp}
\subsection{}
AC is more easily distributed than DC because ...  \\  
a) AC voltage can be stepped up/down with transformers. \\  
b) Wires have less resistance for AC. \\  
c) DC needs insulated wires. \\  
d) AC uses lighter wires.  
\end{prp}

\begin{soln}
\textbf{Solution:}  
Transformers enable efficient voltage adjustment for AC.  
\\ \textbf{Answer:} a) AC voltage can be stepped up and down with transformers.  
\end{soln}

\bigskip
\centerline{\rule{10cm}{0.4pt}}
\bigskip

\begin{prp}
\subsection{}
What quantity increases in the secondary coil of an ideal step-down transformer?
\\
a) Voltage \\  
b) Power \\  
c) Current \\  
d) Frequency  
\end{prp}

\begin{soln}
\textbf{Solution:}  
Step-down transformers reduce voltage but increase current (power conserved).  
\\ \textbf{Answer:} c) Current.  
\end{soln}

\bigskip
\centerline{\rule{10cm}{0.4pt}}
\bigskip

\begin{prp}
\subsection{}
A shunt generator delivers 450 A at 230 V. The resistance of the shunt field and armature are 50 Ω and 0.03 Ω respectively. The generated emf is:\\ 
A- 423.6 V \\  
B- 342.6 V \\  
C- 243.6 V \\  
D- 324.6 V  
\end{prp}

\begin{soln}
\textbf{Solution:}  
1. Armature Voltage Drop:  
\[
V_{\text{armature}} = I \cdot R_{\text{armature}} = 450 \cdot 0.03 = 13.5 \, \text{V}
\]  
2. Shunt Field Current:  
\[
I_{\text{shunt}} = \frac{V}{R_{\text{shunt}}} = \frac{230}{50} = 4.6 \, \text{A}
\]  
3. Generated EMF:  
\[
\mathcal{E} = V + V_{\text{armature}} = 230 + 13.5 = 243.5 \, \text{V}
\]  
\\ \textbf{Answer:} C- 243.6 V  
\end{soln}

\bigskip
\centerline{\rule{10cm}{0.4pt}}
\bigskip

\begin{prp}
\subsection{}
A heater is rated as 230 V, 10 kW, AC. The value 230 V refers to:  \\
A- Peak voltage. \\  
B- R.M.S voltage. \\  
C- Average voltage. \\  
D- None of the above.  
\end{prp}

\begin{soln}
\textbf{Solution:}  
Standard AC voltage ratings (e.g., household supplies) use RMS values.  
\\ \textbf{Answer:} B- R.M.S voltage.  
\end{soln}

\bigskip
\centerline{\rule{10cm}{0.4pt}}
\bigskip

\begin{prp}
\subsection{}
An alternating voltage of maximum value 100 V is applied to a lamp. Which direct voltage would cause the lamp to light with the same brilliance?  \\
A- 63.7 V \\  
B- 70.7 V \\  
C- 100 V \\  
D- 141.4 V  
\end{prp}

\begin{soln}
\textbf{Solution:}  
Equivalent DC voltage = RMS voltage:  
\[
V_{\text{RMS}} = \frac{V_{\text{peak}}}{\sqrt{2}} = \frac{100}{\sqrt{2}} \approx 70.7 \, \text{V}.
\]  
\\ \textbf{Answer:} B- 70.7 V.  
\end{soln}

\bigskip
\centerline{\rule{10cm}{0.4pt}}
\bigskip

\begin{prp}
\subsection{}
A current-carrying coil is subjected to a uniform magnetic field. The coil will orient so that its plane becomes: \\
A- Parallel to the magnetic field \\  
B- Perpendicular to the magnetic field \\  
C- Inclined at 45° to the magnetic field \\  
D- Inclined at any arbitrary angle  
\end{prp}

\begin{soln}

\\ \textbf{Answer:} B- Perpendicular to the magnetic field.  
\end{soln}

\bigskip
\centerline{\rule{10cm}{0.4pt}}
\bigskip

\begin{prp}
\subsection{}
For a DC motor operating under maximum power transfer, the efficiency is: \\
A- About 90\% \\  
B- About 100\% \\  
C- More than 75\% \\  
D- Less than 50\%  
\end{prp}

\begin{soln}
\textbf{Solution:}  
At maximum power transfer, efficiency is 50\% (half power lost internally).  
\\ \textbf{Answer:} D- Less than 50\%.  
\end{soln}

\bigskip
\centerline{\rule{10cm}{0.4pt}}
\bigskip

\begin{prp}
\subsection{}
With increased speed in a DC motor: \\ 
A- Back EMF falls, line current increases \\  
B- Back EMF increases, line current falls \\  
C- Both back EMF and line current fall \\  
D- Both back EMF and line current increase  
\end{prp}

\begin{soln}
\textbf{Solution:}  
Higher speed increases back EMF, reducing line current.  
\\ \textbf{Answer:} B- Back EMF increases but line current falls.  
\end{soln}

\bigskip
\centerline{\rule{10cm}{0.4pt}}
\bigskip

\begin{prp}
\subsection{}
Reversing DC motor direction requires:  
.  
A- Reducing field flux \\  
B- Extra resistance in armature \\  
C- Reversing both armature and field connections \\  
D- Reversing either armature or field connections  
\end{prp}

\begin{soln}
\textbf{Solution:}  
Reverse either armature or field winding connections.  
\\ \textbf{Answer:} D- Reversing connections of either armature or field.  
\end{soln}

\bigskip
\centerline{\rule{10cm}{0.4pt}}
\bigskip

% Unit 11: Transformers (Ph.2.11)
\section{Transformers and Mutual Induction (Ph.2.11).}

\begin{prp}
\subsection{}
Two ideal transformers are designed for 220 V. One steps up to 1200 V, the other down to 3 V. Which delivers more power?  
.  
\\ a) Step-up \\  
b) Step-down \\  
c) Both same \\  
d) Undetermined  
\end{prp}

\begin{soln}
\textbf{Solution:}  
Ideal transformers conserve power (\( P_{\text{primary}} = P_{\text{secondary}} \)).  
\\ \textbf{Answer:} c) Both are the same.  
\end{soln}

\bigskip
\centerline{\rule{10cm}{0.4pt}}
\bigskip

\begin{prp}
\subsection{}
An ideal transformer has 50 secondary turns and 400 primary turns. If 120 V AC is applied to the primary, the output voltage is ...  
.  
\\ a) 8 V \\  
b) 15 V \\  
c) 60 V \\  
d) 240 V  
\end{prp}

\begin{soln}
\textbf{Solution:}  
\[
\frac{V_s}{V_p} = \frac{N_s}{N_p} \Rightarrow V_s = 120 \cdot \frac{50}{400} = 15 \, \text{V}.  
\]
\\ \textbf{Answer:} b) 15 V.  
\end{soln}

\bigskip
\centerline{\rule{10cm}{0.4pt}}
\bigskip

\begin{prp}
\subsection{}
An emf of 5 mV is induced in a coil when the current in a nearby coil changes by 5 A in 0.1 s. The coefficient of mutual induction between the two coils is: \\
A- 1 H \\  
B- 0.1 H \\  
C- 0.1 mH \\  
D- 0.001 mH  
\end{prp}

\begin{soln}
\textbf{Solution:}  
\[
M = \frac{\mathcal{E}}{\Delta I / \Delta t} = \frac{0.005 \, \text{V}}{5 \, \text{A} / 0.1 \, \text{s}} = \frac{0.005}{50} = 0.0001 \, \text{H} = 0.1 \, \text{mH}.
\]  
\\ \textbf{Answer:} C- 0.1 mH.  
\end{soln}

\bigskip
\centerline{\rule{10cm}{0.4pt}}
\bigskip

\begin{prp}
\subsection{}
The coefficient of mutual inductance between two coils depends on: \\ 
A- Separation between the coils \\  
B- Orientation of the coils \\  
C- Medium between the coils \\  
D- All of the above  
\end{prp}

\begin{soln}
\textbf{Solution:}  
Mutual inductance depends on separation, orientation, and medium.  
\\ \textbf{Answer:} D- All of the above.  
\end{soln}

\bigskip
\centerline{\rule{10cm}{0.4pt}}
\bigskip

\begin{prp}
\subsection{}
The plane in which eddy currents are produced in a conductor is inclined to the plane of the magnetic field at an angle of: \\
A- 45° \\  
B- 90° \\  
C- 135° \\  
D- 180°  
\end{prp}

\begin{soln}
\textbf{Solution:}  
Eddy currents form in planes  perpendicular  to the magnetic field.  
\\ \textbf{Answer:} B- 90°.  
\end{soln}

\bigskip
\centerline{\rule{10cm}{0.4pt}}
\bigskip

\begin{prp}
\subsection{}
A step-up transformer with a turns ratio (primary:secondary) of 2:100 operates at 270 V. If the secondary supplies a 3 A load, the primary current is: \\
A- 0.16 A \\  
B- 25 A \\  
C- 100 A \\  
D- 150 A  
\end{prp}

\begin{soln}
\textbf{Solution:}  
\[
\frac{I_p}{I_s} = \frac{N_s}{N_p} \Rightarrow I_p = I_s \cdot \frac{N_s}{N_p} = 3 \cdot \frac{100}{2} = 150 \, \text{A}.
\]  
Answer: D- 150 A.  
\end{soln}

\bigskip
\centerline{\rule{10cm}{0.4pt}}
\bigskip

\begin{prp}
\subsection{}
Maximum mutual inductance between 5 H and 20 H coils (no flux leakage):  \\
A- 2.5 H \\  
B- 5.0 H \\  
C- 7.5 H \\  
D- 10.0 H  
\end{prp}

\begin{soln}
\textbf{Solution:}  
\[
M_{\text{max}} = \sqrt{L_1 L_2} = \sqrt{5 \cdot 20} = 10 \, \text{H}.
\]  
\\ \textbf{Answer:} D- 10.0 H.  
\end{soln}

\bigskip
\centerline{\rule{10cm}{0.4pt}}
\bigskip

\begin{prp}
\subsection{}
Losses in DC generator windings include:  
1) Armature Copper Loss, 2) Rotational Loss, 3) Core Losses.  \\
A- Only 1 \\  
B- 1 \& 2 \\  
C- 2 \& 3 \\  
D- 1, 2 \& 3  
\end{prp}

\begin{soln}
\textbf{Solution:}  
All three losses occur in DC generators.  
\\ \textbf{Answer:} D- 1, 2 \& 3.  
\end{soln}

\bigskip
\centerline{\rule{10cm}{0.4pt}}
\bigskip

\begin{prp}
\subsection{}
A transformer (1600 primary turns, 1200 secondary turns, 6 A primary current) has secondary current: \\
A- 5.48 A \\  
B- 6.28 A \\  
C- 7.18 A \\  
D- 7.98 A  
\end{prp}

\begin{soln}
\textbf{Solution:}  
\[
\frac{N_p}{N_s} = \frac{I_s}{I_p} \Rightarrow I_s = \frac{1600}{1200} \cdot 6 = 8 \, \text{A}.
\]  
 Note:  Options likely contain typos.  
\end{soln}

\bigskip
\centerline{\rule{10cm}{0.4pt}}
\bigskip

\begin{prp}
\subsection{}
Device not using mutual induction:  \\
A- Motor \\  
B- Transformer \\  
C- Tesla Coil \\  
D- Induction Coil  
\end{prp}

\begin{soln}
\textbf{Solution:}  
Motors use electromagnetic induction, not mutual induction.  
\\ \textbf{Answer:} A- Motor.  
\end{soln}

\bigskip
\centerline{\rule{10cm}{0.4pt}}
\bigskip

\begin{prp}
\subsection{}
During transformer open-circuit test:  \\
A- Primary supplied at high voltage \\  
B- Primary supplied with no-load current \\  
C- Secondary supplied rated KVA \\  
D- Primary supplied rated voltage  
\end{prp}

\begin{soln}
\textbf{Solution:}  
Open-circuit test applies  rated voltage  to primary.  
\\ \textbf{Answer:} D- Primary supplied rated voltage.  
\end{soln}

\bigskip
\centerline{\rule{10cm}{0.4pt}}
\bigskip

\begin{prp}
\subsection{}
Eddy current loss at 200 V, 50 Hz (original 240 W at 160 V, 40 Hz):  \\
A- 400 W \\  
B- 275 W \\  
C- 375 W \\  
D- 375 W  
\end{prp}

\begin{soln}
\textbf{Solution:}  
\[
P_{\text{new}} = P_{\text{old}} \cdot \left(\frac{V_{\text{new}}}{V_{\text{old}}}\right)^2 = 240 \cdot \left(\frac{200}{160}\right)^2 = 375 \, \text{W}.
\]  
\\ \textbf{Answer:} C- 375 W.  
\end{soln}

\bigskip
\centerline{\rule{10cm}{0.4pt}}
\bigskip
\section{AC circuits 
containing 
resistive elements(Ph.2.12)}
\begin{prp}
\subsection{}
Hot wire ammeter reads:  \\
A- Peak value \\  
B- R.M.S value \\  
C- Average value \\  
D- None of the above  
\end{prp}

\begin{soln}
\textbf{Solution:}  
Hot wire ammeters measure the  heating effect  of current, which corresponds to the RMS value.  
\\ \textbf{Answer:} B- R.M.S value.  
\end{soln}

\bigskip
\centerline{\rule{10cm}{0.4pt}}
\bigskip

\begin{prp}
\subsection{}
In an AC circuit, the resistance is 4 Ω and the capacitive reactance is 15 Ω. The impedance is:  \\
A- 11.5 Ω \\  
B- 15.5 Ω \\  
C- 19.5 Ω \\  
D- 60.5 Ω  
\end{prp}

\begin{soln}
\textbf{Solution:}  
\[
Z = \sqrt{R^2 + X_C^2} = \sqrt{4^2 + 15^2} = \sqrt{16 + 225} = \sqrt{241} \approx 15.5 \, \Omega.
\]  
\\ \textbf{Answer:} B- 15.5 Ω.  
\end{soln}

\bigskip
\centerline{\rule{10cm}{0.4pt}}
\bigskip

\begin{prp}
\subsection{}
The power curve for a purely resistive circuit is zero only when:  \\
A- Voltage is zero \\  
B- Current is zero \\  
C- Both current and voltage are zero \\  
D- None of these  
\end{prp}

\begin{soln}
\textbf{Solution:}  
In a purely resistive AC circuit, power \( P = V_{\text{rms}}I_{\text{rms}} \cos\phi \). Since \(\phi = 0\), power is zero only when  both \( V \) and \( I \) are zero  simultaneously.  
\\ \textbf{Answer:} C- Both current and voltage are zero.  
\end{soln}

\bigskip
\centerline{\rule{10cm}{0.4pt}}
\bigskip

\begin{prp}
\subsection{}
When are the voltage and current in an LCR-series AC circuit in phase?  \\
A- \( X_L < X_C \) \\  
B- \( X_L > X_C \) \\  
C- \( X_L = X_C \) \\  
D- Indeterminant  
\end{prp}

\begin{soln}
\textbf{Solution:}  
At resonance (\(X_L = X_C\)), the impedance is purely resistive, and voltage/current are in phase.  
\\ \textbf{Answer:} C- \( X_L = X_C \).  
\end{soln}

\bigskip
\centerline{\rule{10cm}{0.4pt}}
\bigskip


\begin{prp}
\subsection{}
A hot wire ammeter reads 20 A in an AC circuit. The peak value of the current is:  
.  
A- \(\frac{20}{\sqrt{2}} \, \text{A}\) \\  
B- \(20\sqrt{2} \, \text{A}\) \\  
C- \(5\pi \, \text{A}\) \\  
D- \(\frac{20}{\pi} \, \text{A}\)  
\end{prp}

\begin{soln}
\textbf{Solution:}  
Hot wire ammeters measure  RMS current . Peak current is:  
\[
I_{\text{peak}} = I_{\text{RMS}} \cdot \sqrt{2} = 20 \cdot \sqrt{2} \, \text{A}.
\]  
\\ \textbf{Answer:} B- \(20\sqrt{2} \, \text{A}\).  
\end{soln}

\bigskip
\centerline{\rule{10cm}{0.4pt}}
\bigskip

\begin{prp}
\subsection{}
Capacitive reactance is more when:  \\

A- Capacitance is less and frequency of supply is less.
B- Capacitance is less and frequency of supply is more \\  
C- Capacitance is more and frequency of supply is less \\  
D- Capacitance is more and frequency of supply is more  
\end{prp}

\begin{soln}  
\\ \textbf{Answer:} A- Capacitance is less and frequency of supply is less.  
\end{soln}

\bigskip
\centerline{\rule{10cm}{0.4pt}}
\bigskip

\begin{prp}
\subsection{}
A pure capacitor of capacitance \(C\) is connected to an AC supply of peak voltage \(V_0\). Which pair of expressions represents the voltage and current?  \\
A- \(V = V_0 \sin \omega t\), \(I = I_0 \sin \omega t\) \\  
B- \(V = V_0 \sin \omega t\), \(I = I_0 \cos \omega t\) \\  
C- \(V = V_0 \sin (\omega t + \pi^2)\), \(I = I_0 \sin \omega t\) \\  
D- \(V = V_0 \sin (\omega t + \pi^2)\), \(I = I_0 \cos \omega t\)  
\end{prp}

\begin{soln}
\textbf{Solution:}  
In a capacitor,  current leads voltage by 90°  (\(\pi/2\)). Voltage: \(V = V_0 \sin \omega t\), Current: \(I = I_0 \cos \omega t\).  
\\ \textbf{Answer:} B- \(V = V_0 \sin \omega t\), \(I = I_0 \cos \omega t\).  
\end{soln}

\bigskip
\centerline{\rule{10cm}{0.4pt}}
\bigskip

\begin{prp}
\subsection{}
The maximum value of alternating voltage is given by:  \\
A- Coefficient of sine term \\  
B- Frequency \\  
C- Alternation \\  
D- Negative amplitude  
\end{prp}

\begin{soln}
\textbf{Solution:}  
Peak voltage is the coefficient of the sine function.  
\\ \textbf{Answer:} A- Coefficient of sine term.  
\end{soln}

\bigskip
\centerline{\rule{10cm}{0.4pt}}
\bigskip

\section{Characteristics of Circuits (Ph.2.13)}

\begin{prp}
\subsection{}
In a series LCR-circuit, the capacitive reactance is 120 Ω, inductive reactance is 200 Ω, and resistance is 60 Ω. The impedance is:  \\
A- 100 Ω \\  
B- 120 Ω \\  
C- 200 Ω \\  
D- 250 Ω  
\end{prp}

\begin{soln}
\textbf{Solution:}  
\[
Z = \sqrt{R^2 + (X_L - X_C)^2} = \sqrt{60^2 + (200 - 120)^2} = \sqrt{3600 + 6400} = 100 \, \Omega.
\]  
\\ \textbf{Answer:} A- 100 Ω.  
\end{soln}

\bigskip
\centerline{\rule{10cm}{0.4pt}}
\bigskip

\begin{prp}
\subsection{}
A series resonant band-pass filter has a 2 mH coil (15 Ω winding resistance), 0.005 μF capacitor, and 150 Ω resistor. At center frequency (\(f_0\)), the input voltage is 20V, so the output voltage magnitude is:  
.  
A- 10.9 V \\  
B- 18.75 V \\  
C- 19.1 V \\  
D- 20 V  
\end{prp}

\begin{soln}
\textbf{Solution:}  
At resonance:  
\[
Z_{\text{total}} = R_{\text{resistor}} + R_{\text{coil}}} = 150 + 15 = 165 \, \Omega.
\]  
\[
I = \frac{V_{\text{in}}}{Z_{\text{total}}} = \frac{20}{165} \approx 0.1212 \, \text{A}.
\]  
\[
V_{\text{out}}} = I \cdot R_{\text{resistor}}} = 0.1212 \times 150 \approx 18.18 \, \text{V}.
\]  
 Closest answer:  B- 18.75 V.  
\end{soln}

\bigskip
\centerline{\rule{10cm}{0.4pt}}
\bigskip

\begin{prp}
\subsection{}
An RC low-pass filter has \( R = 8.2 \, \Omega \) and \( C = 0.0033 \, \mu\text{F} \). The cutoff frequency is:  
.  
A- 7.00 MHz \\  
B- 5.88 MHz \\  
C- 4.26 kHz \\  
D- 6.00 kHz  
\end{prp}

\begin{soln}
\textbf{Solution:}  
\[
f_c = \frac{1}{2\pi RC} = \frac{1}{2\pi \cdot 8.2 \cdot 0.0033 \times 10^{-6}} \approx 5.88 \, \text{MHz}.
\]  
\\ \textbf{Answer:} B- 5.88 MHz.  
\end{soln}

\bigskip
\centerline{\rule{10cm}{0.4pt}}
\bigskip

\begin{prp}
\subsection{}
The input frequency applied to a low-pass filter is:  
.  
A- Zero \\  
B- High \\  
C- Low \\  
D- Constant  
\end{prp}

\begin{soln}
\textbf{Solution:}  
Jun 1, 2024 — The input frequency applied to a low pass filter is typically low, allowing signals with frequencies lower than a certain cutoff frequency.  
\\ \textbf{Answer:} C- Low.  
\end{soln}

\bigskip
\centerline{\rule{10cm}{0.4pt}}
\bigskip

\begin{prp}
\subsection{}
Which component is part of the input in a band-pass filter?  
.  
A- Resistor \\  
B- Capacitor \\  
C- Inductor \\  
D- Both A and B  
\end{prp}

\begin{soln}
\textbf{Solution:}  
Band-pass filters often use combinations of  resistors and capacitors  (RC/LC networks) at the input.  
\\ \textbf{Answer:} D- Both A and B.  
\end{soln}

\bigskip
\centerline{\rule{10cm}{0.4pt}}
\bigskip

\begin{prp}
\subsection{}
A 200 V AC source is fed to a series LCR circuit with \( X_L = 50 \, \Omega \), \( X_C = 50 \, \Omega \), and \( R = 25 \, \Omega \). The potential drop across the inductor is:  
.  
A- 100 V \\  
B- 200 V \\  
C- 300 V \\  
D- 400 V  
\end{prp}

\begin{soln}
\textbf{Solution:}  
At resonance (\( X_L = X_C \)), impedance \( Z = R = 25 \, \Omega \). Current:  
\[
I = \frac{V}{Z} = \frac{200}{25} = 8 \, \text{A}.
\]  
Voltage across inductor:  
\[
V_L = I \cdot X_L = 8 \cdot 50 = 400 \, \text{V}.
\]  
\\ \textbf{Answer:} D- 400 V.  
\end{soln}

\bigskip
\centerline{\rule{10cm}{0.4pt}}
\bigskip

\begin{prp}
\subsection{}
In an LCR circuit at resonance:  
.  
A- Current is minimum \\  
B- Impedance is maximum \\  
C- Current leads voltage by \( \pi/2 \) \\  
D- Current and voltage are in phase  
\end{prp}

\begin{soln}
\textbf{Solution:}  
At resonance, impedance is  minimum , current is  maximum , and voltage/current are  in phase .  
\\ \textbf{Answer:} D- The current and voltage are in the same phase.  
\end{soln}

\bigskip
\centerline{\rule{10cm}{0.4pt}}
\bigskip

\begin{prp}
\subsection{}
A series resonant band-pass filter (2 mH coil, 0.005 μF capacitor, 150 Ω resistor, coil resistance 15 Ω) has an input of 20 V rms. Output voltage at center frequency is:  
.  
A- 10.9 V \\  
B- 18.75 V \\  
C- 19.1 V \\  
D- 20 V  
\end{prp}

\begin{soln}
\textbf{Solution:}  
\[
Z_{\text{total}} = R_{\text{resistor}} + R_{\text{coil}} = 150 + 15 = 165 \, \Omega, \quad I = \frac{20}{165} \approx 0.1212 \, \text{A}, \quad V_{\text{out}} = 0.1212 \cdot 150 \approx 18.18 \, \text{V}.
\]  
\\ \textbf{Answer:} B- 18.75 V.  
\end{soln}

\bigskip
\centerline{\rule{10cm}{0.4pt}}
\bigskip

\begin{prp}
\subsection{}
A low-pass filter is basically:  
.  
A- Differentiating circuit with low time constant \\  
B- Differentiating circuit with larger time constant \\  
C- Integrating circuit with low time constant \\  
D- Integrating circuit with large time constant  
\end{prp}

\begin{soln}
\textbf{Solution:}  
Low-pass filters act as  integrating circuits with large time constants .  
\\ \textbf{Answer:} D- Integrating circuit with large time constant.  
\end{soln}

\bigskip
\centerline{\rule{10cm}{0.4pt}}
\bigskip

\begin{prp}
\subsection{}
In a series LCR-circuit (\(X_C = 120 \, \Omega\), \(X_L = 200 \, \Omega\), \(R = 60 \, \Omega\)), the impedance is:  
.  
A- 100 Ω \\  
B- 120 Ω \\  
C- 200 Ω \\  
D- 250 Ω  
\end{prp}

\begin{soln}
\textbf{Solution:}  
\[
Z = \sqrt{R^2 + (X_L - X_C)^2} = \sqrt{60^2 + 80^2} = 100 \, \Omega.
\]  
\\ \textbf{Answer:} A- 100 Ω.  
\end{soln}

\bigskip
\centerline{\rule{10cm}{0.4pt}}
\bigskip

% Unit 14: Semiconductors (Ph.2.14)
\section{Semiconductors and Energy Bands (Ph.2.14).}

\begin{prp}
\subsection{}
To make P-type semiconductors, ...  
.  
\\ a) Dope with acceptor impurity. \\  
b) Dope with donor impurity. \\  
c) Inject electrons. \\  
d) Inject holes.  
\end{prp}

\begin{soln}
\textbf{Solution:}  
P-type uses acceptors (e.g., boron in silicon).  
\\ \textbf{Answer:} \\ a) A pure material is doped with an acceptor impurity.  
\end{soln}


\bigskip
\centerline{\rule{10cm}{0.4pt}}
\bigskip

\begin{prp}
\subsection{}
When a pure silicon crystal is doped with phosphorus atoms, ...  
.  
\\ a) Resistivity increases. \\  
b) Conductivity increases. \\  
c) Holes exceed electrons. \\  
d) Crystal becomes negative.  
\end{prp}

\begin{soln}
\textbf{Solution:}  
Phosphorus adds free electrons (n-type), increasing conductivity.  
\\ \textbf{Answer:} d) Its electric conductivity gets higher.  
\end{soln}

\bigskip
\centerline{\rule{10cm}{0.4pt}}
\bigskip

\begin{prp}
\subsection{}
Donor impurity atoms in semiconducting material result in a new:  
.  
A- Wide energy band \\  
B- Narrow energy band \\  
C- Discrete energy level just above valence band \\  
D- Discrete energy level just below conduction band  
\end{prp}

\begin{soln}
\textbf{Solution:}  
Donor impurities (e.g., phosphorus in silicon) introduce a discrete energy level  just below the conduction band .  
\\ \textbf{Answer:} D- Discrete energy level just below conduction band.  
\end{soln}

\bigskip
\centerline{\rule{10cm}{0.4pt}}
\bigskip

\begin{prp}
\subsection{}
The electrical conductivity of a semiconductor increases when radiation of wavelength shorter than 1000 nm is incident. The band gap is:  

A- 2.4 eV \\  
B- 2.2 eV \\  
C- 1.2 eV \\  
D- 0.8 eV  
\end{prp}

\begin{soln}
\textbf{Solution:}  
Band gap energy \(E_g = \frac{hc}{\lambda}\). For \(\lambda < 1000 \, \text{nm}\):  
\[
E_g > \frac{1240 \, \text{eV·nm}}{1000 \, \text{nm}} = 1.24 \, \text{eV}.  
\]  
Closest option: C- 1.2 eV.  
\\ \textbf{Answer:} C- 1.2 eV.  
\end{soln}

\bigskip
\centerline{\rule{10cm}{0.4pt}}
\bigskip

\begin{prp}
\subsection{}
An electron in the conduction band:  
 
A- Has no charge \\  
B- Is bound to its parent atom \\  
C- Is located near the top of the crystal \\  
D- Has higher energy than an electron in the valence band  
\end{prp}

\begin{soln}
\textbf{Solution:}  
Conduction band electrons are  free to move  and have  higher energy  than valence band electrons.  
\\ \textbf{Answer:} D- Has a higher energy than an electron in the valence band.  
\end{soln}

\bigskip
\centerline{\rule{10cm}{0.4pt}}
\bigskip

\begin{prp}
\subsection{}
At 0 K, the conduction band may be partially filled in:  
A- Insulators only \\  
B- Conductors only \\  
C- Semiconductors only \\  
D- Conductors and semiconductors  
\end{prp}

\begin{soln}
\textbf{Solution:}  
In  conductors , the conduction band is partially filled even at 0 K. Semiconductors/insulators have empty conduction bands.  
\\ \textbf{Answer:} B- Conductors only.  
\end{soln}

\bigskip
\centerline{\rule{10cm}{0.4pt}}
\bigskip

\begin{prp}
\subsection{}
 Assertion (\\ a):  Conductivity of semiconductors increases on doping.  
 Reason (R):  Doping always increases the number of electrons.  
A- Both A and R are true, and R explains A \\  
B- Both A and R are true, but R does not explain A \\  
C- A is true, R is false \\  
D- A and R are false  
\end{prp}

\begin{soln}
\textbf{Solution:}  
Doping can increase  electrons (n-type)  or  holes (p-type) . R is false because doping doesn’t always add electrons.  
\\ \textbf{Answer:} C- A is TRUE but R is FALSE.  
\end{soln}

\bigskip
\centerline{\rule{10cm}{0.4pt}}
\bigskip

\begin{prp}
\subsection{}
Pure metals generally have:  

A- High conductivity \& low temperature coefficient \\  
B- High conductivity \& high temperature coefficient \\  
C- Low conductivity \& zero temperature coefficient \\  
D- Low conductivity \& high temperature coefficient  
\end{prp}

\begin{soln}
\textbf{Solution:}  
Metals have  high conductivity  and  positive temperature coefficient  (resistance increases with temperature).  
\\ \textbf{Answer:} B- High conductivity \& high temperature coefficient.  
\end{soln}

\bigskip
\centerline{\rule{10cm}{0.4pt}}
\bigskip

\begin{prp}
\subsection{}
The cutoff wavelengths for Si (1.1 eV) and Ge (0.67 eV) photodiodes are:  
A- 1850.27 nm, 2167.91 nm \\  
B- 456.12 nm, 1127.27 nm \\  
C- 1315.45 nm, 1850.75 nm \\  
D- 1127.27 nm, 1850.75 nm  
\end{prp}

\begin{soln}
\textbf{Solution:}  
\[
\lambda = \frac{hc}{E_g} = \frac{1240 \, \text{eV·nm}}{E_g \, (\text{eV})}.  
\]  
- Si: \(\lambda = 1127.27 \, \text{nm}\)  
- Ge: \(\lambda = 1850.75 \, \text{nm}\).  
\\ \textbf{Answer:} D- 1127.27 nm, 1850.75 nm.  
\end{soln}

\bigskip
\centerline{\rule{10cm}{0.4pt}}
\bigskip

\begin{prp}
\subsection{}
An intrinsic semiconductor at absolute zero behaves as:  

A- An insulator \\  
B- A semiconductor \\  
C- A superconductor \\  
D- A metallic conductor  
\end{prp}

\begin{soln}
\textbf{Solution:}  
At 0 K, intrinsic semiconductors have no free carriers \(\Rightarrow\) behave as insulators.  
\\ \textbf{Answer:} A- An insulator.  
\end{soln}

\bigskip
\centerline{\rule{10cm}{0.4pt}}
\bigskip

\begin{prp}
\subsection{}
A semiconductor doped with ppm-level impurities becomes:  
A- Extrinsic semiconductor \\  
B- Intrinsic semiconductor \\  
C- Elemental semiconductor \\  
D- Compound semiconductor  
\end{prp}

\begin{soln}
\textbf{Solution:}  
Doped semiconductors are called  extrinsic .  
\\ \textbf{Answer:} A- Extrinsic semiconductor.  
\end{soln}

\bigskip
\centerline{\rule{10cm}{0.4pt}}
\bigskip


\begin{prp}
\subsection{}
In a PN junction with no external voltage, the electric field between acceptor and donor ions is called a:  
A- Path \\  
B- Peak \\  
C- Barrier \\  
D- Threshold  
\end{prp}

\begin{soln}
\textbf{Solution:}  
The built-in electric field is termed the  barrier potential .  
\\ \textbf{Answer:} C- Barrier.  
\end{soln}

\bigskip
\centerline{\rule{10cm}{0.4pt}}
\bigskip

\begin{prp}
\subsection{}
A hole refers to:  

A- A proton \\  
B- A positively charged electron \\  
C- An electron that lost its charge \\  
D- A microscopic defect \\  
E- Absence of an electron in a filled band  
\end{prp}

\begin{soln}
\textbf{Solution:}  
A hole represents the  absence of an electron  in the valence band.  
\\ \textbf{Answer:} E- Absence of an electron in an otherwise filled band.  
\end{soln}

\bigskip
\centerline{\rule{10cm}{0.4pt}}
\bigskip

\begin{prp}
\subsection{}
Donor atoms introduced into a pure semiconductor at room temperature:  

A- Increase conduction band electrons \\  
B- Increase valence band holes \\  
C- Lower Fermi level \\  
D- Increase resistivity \\  
E- None of the above  
\end{prp}

\begin{soln}
\textbf{Solution:}  
Donor atoms (e.g., phosphorus) donate free electrons to the conduction band.  
\\ \textbf{Answer:} A- Increase the number of electrons in the conduction band.  
\end{soln}

\bigskip
\centerline{\rule{10cm}{0.4pt}}
\bigskip

\begin{prp}
\subsection{}
Acceptor atoms introduced into a pure semiconductor:  

A- Increase conduction band electrons \\  
B- Increase valence band holes \\  
C- Raise Fermi level \\  
D- Increase resistivity \\  
E- None of the above  
\end{prp}

\begin{soln}
\textbf{Solution:}  
Acceptor atoms (e.g., boron) create holes in the valence band.  
\\ \textbf{Answer:} B- Increase the number of holes in the valence band.  
\end{soln}

\bigskip
\centerline{\rule{10cm}{0.4pt}}
\bigskip

\begin{prp}
\subsection{}
An acceptor atom in silicon has how many outer shell electrons?  

A- 3 \\  
B- 4 \\  
C- 5 \\  
D- 6 \\  
E- 7  
\end{prp}

\begin{soln}
\textbf{Solution:}  
Acceptor atoms (Group III) have  3 valence electrons  (e.g., boron).  
\\ \textbf{Answer:} A- 3.  
\end{soln}

\bigskip
\centerline{\rule{10cm}{0.4pt}}
\bigskip

\begin{prp}
\subsection{}
A donor atom in silicon has how many outer shell electrons?  

A- 1 \\  
B- 2 \\  
C- 3 \\  
D- 4 \\  
E- 5  
\end{prp}

\begin{soln}
\textbf{Solution:}  
Donor atoms (Group V) have  5 valence electrons  (e.g., phosphorus).  
\\ \textbf{Answer:} E- 5.  
\end{soln}

\bigskip
\centerline{\rule{10cm}{0.4pt}}
\bigskip

\begin{prp}
\subsection{}
Application of forward bias to a p-n junction:  
A- Narrows depletion zone \\  
B- Increases electric field \\  
C- Increases potential difference \\  
D- Increases n-side donors \\  
E- Decreases n-side donors  
\end{prp}

\begin{soln}
\textbf{Solution:}  
Forward bias reduces depletion zone width.  
\\ \textbf{Answer:} A- Narrows the depletion zone.  
\end{soln}

\bigskip
\centerline{\rule{10cm}{0.4pt}}
\bigskip

\begin{prp}
\subsection{}
Forward bias increases:  

A- Drift current \\  
B- Diffusion current \\  
C- Decreases p-side drift \\  
D- Decreases n-side drift \\  
E- No current change  
\end{prp}

\begin{soln}
\textbf{Solution:}  
Forward bias enhances  diffusion current  across the junction.  
\\ \textbf{Answer:} B- Increases diffusion current.  
\end{soln}

\bigskip
\centerline{\rule{10cm}{0.4pt}}
\bigskip

\begin{prp}
\subsection{}
With forward bias, electron concentration on the p-side:  

A- Slightly increases \\  
B- Dramatically increases \\  
C- Slightly decreases \\  
D- Dramatically decreases \\  
E- No change  
\end{prp}

\begin{soln}
\textbf{Solution:}  
Forward bias injects electrons into the p-side, significantly increasing their concentration.  
\textbf{Answer:} B- Increases dramatically.  
\end{soln}

\bigskip
\centerline{\rule{10cm}{0.4pt}}
\bigskip


% Unit 15: Diodes (Ph.2.15)
\section{DC and AC Circuits with Diodes (Ph.2.15).}

\begin{prp}
\subsection{}
When a diode is in forward bias, ...  

\\ a) Diffusion current increases. \\  
b) Drift current increases. \\  
c) Depletion layer thickens. \\  
d) Potential barrier increases.  
\end{prp}

\begin{soln}
\textbf{Solution:}  
Forward bias reduces barrier, allowing diffusion current.  
\textbf{Answer:} \\ a) Diffusion current through the diode increases.  
\end{soln}

\bigskip
\centerline{\rule{10cm}{0.4pt}}
\bigskip

\begin{prp}
\subsection{}
The capacitance of a reverse biased PN junction:  
A- Increases as reverse bias is increased \\  
B- Increases as reverse bias is decreased \\  
C- Decreases as reverse bias is increased \\  
D- Is insignificantly low  
\end{prp}

\begin{soln}
\textbf{Solution:}  
Reverse bias widens the depletion layer, reducing capacitance (\(C \propto \frac{1}{\sqrt{V}}\)).  
\textbf{Answer:} C- Decreases as reverse bias is increased.  
\end{soln}

\bigskip
\centerline{\rule{10cm}{0.4pt}}
\bigskip

\begin{prp}
\subsection{}
The ripple voltage of a full-wave rectifier with a 100 μF filter capacitor and 50 mA load is:  

A- 2.4 kV \\  
B- 4.8 kV \\  
C- 1.2 kV \\  
D- 6.6 kV  
\end{prp}

\begin{soln}
\textbf{Solution:}  
\[
V_{\text{ripple}} = \frac{I_{\text{load}}}{2fC} = \frac{0.05}{2 \cdot 50 \cdot 100 \times 10^{-6}}} = 5 \, \text{V}.
\]  
Note: Options are unrealistic. Closest logic: None match.  
\end{soln}

\bigskip
\centerline{\rule{10cm}{0.4pt}}
\bigskip

\begin{prp}
\subsection{}
Reverse bias current in a PN junction diode is typically:  

A- Few amperes \\  
B- Few milliamperes \\  
C- 0.2–15 A \\  
D- Few micro/nano amperes  
\end{prp}

\begin{soln}
\textbf{Solution:}  
Reverse saturation current is very small (μA/nA range).  
\textbf{Answer:} D- Few micro or nano amperes.  
\end{soln}

\bigskip
\centerline{\rule{10cm}{0.4pt}}
\bigskip

\begin{prp}
\subsection{}
For a PN junction with \(N_A = N_D = 10^{20}/\text{cm}^3\) vs. \(10^{14}/\text{cm}^3\): 
\\
A- Lower breakdown voltage, lower capacitance \\  
B- Higher breakdown voltage, lower capacitance \\  
C- Lower breakdown voltage, higher capacitance \\  
D- Higher breakdown voltage, higher capacitance  
\end{prp}

\begin{soln}    
\textbf{Answer:} c- Lower breakdown voltage and higher capacitance.  
\end{soln}

\bigskip
\centerline{\rule{10cm}{0.4pt}}
\bigskip

% Unit 16: Transistors and Logic Gates (Ph.2.16)
\section{Transistors and Digital Logic (Ph.2.16).}

\begin{prp}
\subsection{}
The output of an AND gate with three inputs (A, B, C) is HIGH when ...  

\\ a) A=1, B=1, C=0 \\  
b) A=0, B=0, C=0 \\  
c) A=1, B=1, C=1 \\  
d) A=1, B=0, C=1  
\end{prp}

\begin{soln}
\textbf{Solution:}  
AND gate requires all inputs HIGH.  
\textbf{Answer:} c) A=1, B=1, C=1.  
\end{soln}

\bigskip
\centerline{\rule{10cm}{0.4pt}}
\bigskip

\begin{prp}
\subsection{}
How many inputs of a four-input OR gate must be HIGH for the output to go HIGH?  
\\ a) Any one \\  
b) Any two \\  
c) Any three \\  
d) All four  
\end{prp}

\begin{soln}
\textbf{Solution:}  
OR gate outputs HIGH if  at least one  input is HIGH.  
\textbf{Answer:} a) Any one of the inputs.  
\end{soln}

\bigskip
\centerline{\rule{10cm}{0.4pt}}
\bigskip

\begin{prp}
\subsection{}
If a 3-input OR gate has eight input possibilities, how many result in HIGH output?  
\\ a) 1 \\  
b) 2 \\  
c) 7 \\  
d) 8  
\end{prp}

\begin{soln}
\textbf{Solution:}  
Only  all zeros  gives LOW; 7 combinations have at least one HIGH.  
\textbf{Answer:} c) 7.  
\end{soln}

\bigskip
\centerline{\rule{10cm}{0.4pt}}
\bigskip

\begin{prp}
\subsection{}
If a transistor’s Q-point is at the middle of the load line, decreasing current gain moves the Q-point:  \\
A- Up \\  
B- Down \\  
C- Nowhere \\  
D- Off the load line  
\end{prp}

\begin{soln}
\textbf{Solution:}  
Lower \(\beta\) reduces collector current, shifting Q-point  downward .  
\textbf{Answer:} B- Down.  
\end{soln}

\bigskip
\centerline{\rule{10cm}{0.4pt}}
\bigskip

\begin{prp}
\subsection{}
The binary equivalent of 25 is:  \\
A- 11010 \\  
B- 10111 \\  
C- 11001 \\  
D- 10101  
\end{prp}

\begin{soln}
\textbf{Solution:}  
\(25_{10} = 11001_2\).  
\textbf{Answer:} C- 11001.  
\end{soln}

\bigskip
\centerline{\rule{10cm}{0.4pt}}
\bigskip

\begin{prp}
\subsection{}
The output is LOW when any input is zero in a(n):  \\
A- OR gate \\  
B- NOT gate \\  
C- AND gate \\  
D- NAND gate  
\end{prp}

\begin{soln}
\textbf{Solution:}  
AND gate outputs HIGH  only if all inputs are HIGH 
\textbf{Answer:} C- AND gate.  
\end{soln}

\bigskip
\centerline{\rule{10cm}{0.4pt}}
\bigskip

\begin{prp}
\subsection{}
Common-base current gain (\(I_c = 4.0 \, \text{mA}, I_e = 4.2 \, \text{mA}\)): \\
A- 0.20 \\  
B- 0.95 \\  
C- 1.05 \\  
D- 16.80  
\end{prp}

\begin{soln}
\textbf{Solution:}  
\[
\alpha = \frac{I_c}{I_e} = \frac{4.0}{4.2} \approx 0.95.
\]  
\textbf{Answer:} B- 0.95.  
\end{soln}

\bigskip
\centerline{\rule{10cm}{0.4pt}}
\bigskip

\begin{prp}
\subsection{}
Reverse saturation current in a diode is independent of: \\
A- Temperature \\  
B- Junction area \\  
C- Potential barrier \\  
D- Doping concentration  
\end{prp}

\begin{soln}
\textbf{Solution:}  
Reverse saturation current depends on  temperature, doping, and area , not the barrier height.  
\textbf{Answer:} C- Potential barrier.  
\end{soln}

\bigskip
\centerline{\rule{10cm}{0.4pt}}
\bigskip

\begin{prp}
\subsection{}
Transistor biasing for amplifier operation requires: \\
A- EB reverse, CB forward \\  
B- EB forward, CB forward \\  
C- EB reverse, CB reverse \\  
D- EB forward, CB reverse  
\end{prp}

\begin{soln}
\textbf{Solution:}  
Active mode:  EB forward ,  CB reverse .  
\textbf{Answer:} D- EB forward, CB reverse.  
\end{soln}

\bigskip
\centerline{\rule{10cm}{0.4pt}}
\bigskip

\begin{prp}
\subsection{}
Common-emitter DC current gain (\(\alpha = 0.9\)): \\
A- 9 \\  
B- 19 \\  
C- 29 \\  
D- 39  
\end{prp}

\begin{soln}
\textbf{Solution:}  
\[
\beta = \frac{\alpha}{1 - \alpha} = \frac{0.9}{0.1} = 9.
\]  
\textbf{Answer:} A- 9.  
\end{soln}

\bigskip
\centerline{\rule{10cm}{0.4pt}}
\bigskip

\begin{prp}
\subsection{}
NOT gate output is HIGH when: \\
A- Input is HIGH \\  
B- Input is LOW \\  
C- Input changes LOW→HIGH \\  
D- Voltage is removed  
\end{prp}

\begin{soln}
\textbf{Solution:}  
NOT gate inverts input: HIGH → LOW, LOW → HIGH.  
\textbf{Answer:} B- Input is LOW.  
\end{soln}

\bigskip
\centerline{\rule{10cm}{0.4pt}}
\bigskip

\begin{prp}
\subsection{}
Valid for both PNP and NPN transistors: \\
A- Emitter injects holes into base \\  
B- Electrons are minority carriers in base \\  
C- EB junction forward biased \\  
D- Current flows into emitter in active region  
\end{prp}

\begin{soln}
\textbf{Solution:}  
For active operation,  EB junction is always forward biased  (true for both types).  
\textbf{Answer:} C- EB junction forward biased.  
\end{soln}

\bigskip
\centerline{\rule{10cm}{0.4pt}}
\bigskip

\begin{prp}
\subsection{}
The current ratio \( \frac{I_C}{I_E} \) is called: \\
A- Beta \\  
B- Theta \\  
C- Alpha \\  
D- Omega  
\end{prp}

\begin{soln}
\textbf{Solution:}  
\( \alpha = \frac{I_C}{I_E} \).  
\textbf{Answer:} C- Alpha.  
\end{soln}

\bigskip
\centerline{\rule{10cm}{0.4pt}}
\bigskip

\begin{prp}
\subsection{}
Beta (\( \beta \)) is the ratio: \\
A- \( \frac{I_C}{I_B} \) \\  
B- \( \frac{I_C}{I_E} \) \\  
C- \( \frac{I_E}{I_B} \) \\  
D- \( \frac{I_E}{I_C} \)  
\end{prp}

\begin{soln}
\textbf{Solution:}  
\( \beta = \frac{I_C}{I_B} \).  
\textbf{Answer:} A- \( \frac{I_C}{I_B} \).  
\end{soln}

\bigskip
\centerline{\rule{10cm}{0.4pt}}
\bigskip

\begin{prp}
\subsection{}
A collector characteristic curve plots: \\
A- \( I_E \) vs \( V_{CE} \) \\  
B- \( I_C \) vs \( V_{CE} \) \\  
C- \( I_C \) vs \( V_{CC} \) \\  
D- \( I_E \) vs \( V_{CC} \)  
\end{prp}

\begin{soln}
\textbf{Solution:}  
Collector current (\( I_C \)) vs \( V_{CE} \) at constant \( V_{BB} \).  
\textbf{Answer:} B- \( I_C \) versus \( V_{CE} \).  
\end{soln}

\bigskip
\centerline{\rule{10cm}{0.4pt}}
\bigskip

\begin{prp}
\subsection{}
For a silicon transistor in C-E configuration, \( V_{BE} \) is: \\
A- Voltage-divider bias \\  
B- 0.3 V \\  
C- 0.7 V \\  
D- 0.0 V  
\end{prp}

\begin{soln}
\textbf{Solution:}  
Silicon transistors require \( V_{BE} \approx 0.7 \, \text{V} \) for forward bias.  
\textbf{Answer:} C- 0.7 V.  
\end{soln}

\bigskip
\centerline{\rule{10cm}{0.4pt}}
\bigskip

\begin{prp}
\subsection{}
In a PNP circuit, the most positive voltage is: \\
A- Ground \\  
B- \( V_C \) \\  
C- \( V_{BE} \) \\  
D- \( V_{CC} \)  
\end{prp}

\begin{soln}
\textbf{Solution:}  
In PNP transistors, the emitter is connected to the highest positive voltage (often ground in some configurations).  
\textbf{Answer:} A- Ground.  
\end{soln}

\bigskip
\centerline{\rule{10cm}{0.4pt}}
\bigskip

\begin{prp}
\subsection{}
Most electrons in the base of an NPN transistor flow: \\
A- Out of the base lead \\  
B- Into the collector \\  
C- Into the emitter \\  
D- Into the base supply  
\end{prp}

\begin{soln}
\textbf{Solution:}  
Electrons from the emitter are collected by the collector.  
\textbf{Answer:} B- Into the collector.  
\end{soln}

\bigskip
\centerline{\rule{10cm}{0.4pt}}
\bigskip

\begin{prp}
\subsection{}
Collector current is controlled by: \\
A- Collector voltage \\  
B- Base current \\  
C- Collector resistance \\  
D- All of the above  
\end{prp}

\begin{soln}
\textbf{Solution:}  
\( I_C \) is proportional to \( I_B \) in active mode.  
\textbf{Answer:} B- Base current.  
\end{soln}

\bigskip
\centerline{\rule{10cm}{0.4pt}}
\bigskip

\begin{prp}
\subsection{}
Total emitter current is: \\
A- \( I_E - I_C \) \\  
B- \( I_C + I_E \) \\  
C- \( I_B + I_C \) \\  
D- \( I_B - I_C \)  
\end{prp}

\begin{soln}
\textbf{Solution:}  
\( I_E = I_B + I_C \).  
\textbf{Answer:} C- \( I_B + I_C \).  
\end{soln}

\HRule \vspace{0.5cm}
\center
\color{black}
{
\< وَفْقَكُمْ اللَّهُ رَبُّ الْعَالَمِينَ >
 \\[0.15cm]}
{ \large \bfseries STEM High School for Boys - 6th of October \\[0.15cm]}
{ \bfseries Omar Yahya - Yaseen Saad-Eldin \\[0.5cm]}
\HRule \\[0.5cm]
\includegraphics[scale=.08]{download.png} \\[.5cm]
\vspace*{\fill}

\end{document}
